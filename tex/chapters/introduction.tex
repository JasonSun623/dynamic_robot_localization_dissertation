\chapter{Introduction} \label{chap:introduction}



\section*{}

This chapter provides an overview about the motivations and objectives of this dissertation along with its practical applications.



\section{Context} \label{sec:introduction_context}

Humanity has sought a reliable method of navigation ever since it started to explore the world. It began with simple landmark reference points for local travels, then perfected celestial navigation for global journeys, and when it finally conquered space, it deployed a global system for high accuracy localization.

Autonomous robots face the same problem, because in order to be able to navigate with precision, they first need to know their location.

Over the years, several localization methods have been proposed and refined, according to the navigation environment and the accuracy requirements. Some are meant for high precision local navigation, while others provide an approximate global position.

A robot capable of operating safely and accurately in a dynamic environment can have innumerous applications, ranging from simple delivery tasks to advanced assembly. Besides improving productivity by performing repetitive tasks with precision and speed, robots can also act as coworkers, helping humans perform their jobs more efficiently and thus, reducing the overall production costs.



\section{Project} \label{sec:introduction_project}

\gls{carlos} is a European research project that aims to develop an autonomous robot capable of performing repetitive tasks alongside human co-workers in dynamic environments.

The robot will operate in shipyards and will be designed to perform fit-out operations, such as stud welding and \gls{cad} drawing.

Stud welding is a repetitive task that provides structural support for other components, such as heat insulation layers or electrical systems.

\gls{cad} drawing will help human co-workers assemble components faster, because it will mark the exact position in which the systems must be installed.

This dissertation aims to provide the localization system for this robot, in order to allow precise navigation in the environment.



\section{Motivation and objectives} \label{sec:introduction_goals}\

With the increase of competitiveness in the current globalized trading markets, companies are trying to reduce production costs and improve the productivity of their assets. One way to achieve these goals relies on robots to do the simple and repetitive jobs while giving humans more free time to perform the more complex and creative tasks.

For these robots be able to navigate autonomously in the environment, they need a precise and efficient localization system. This is critical because the robots must perform their tasks at the correct location, otherwise their work will be useless and may even create additional costs to repair the damage caused.

The main objective of this dissertation is the development of a modular, efficient and reliable localization system for dynamic environments, capable to run in real time on the robots on-board computer systems.



\section{Dissertation outline} \label{sec:introduction_structure}

The remaining of this dissertation is split over 6 chapters. \Cref{chap:localization-methods} provides an overview of the main localization methods that can be used with today's technology. \cref{chap:relevant-sofware-technologies} introduces the frameworks used to build the localization system that is detailed in \cref{chap:localization-system}. \Cref{chap:point-cloud-algorithms} gives the theoretical foundations and algorithms that can be used to process and analyze point clouds. \Cref{chap:localization-system-evaluation} evaluates the results achieved with the localization system on the different testing conditions and environments. Finally, \cref{chap:conclusions-and-future-work} presents the conclusions of this dissertation and suggestions for future work.
