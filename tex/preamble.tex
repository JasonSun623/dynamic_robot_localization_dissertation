\documentclass[11pt,a4paper,twoside,openright]{report}

\usepackage[utf8]{inputenc}
\usepackage[portuguese,english]{babel}
\selectlanguage{english}

\usepackage[mieic]{styles/feupteses}
% Additional options for feupteses.sty:
% - juri: prints line numbers
% - final: final version
% - onpaper: links are not shown (for paper versions)
% - backrefs: include back references from bibliography to citation place

\usepackage[lofdepth,lotdepth]{subfig}
\usepackage{graphicx}
\usepackage{flafter}
\usepackage{animate}
\usepackage{float}
\usepackage{grffile}

\usepackage{color}
\definecolor{cloudwhite}{cmyk}{0,0,0,0.025}

\usepackage{listings}
\lstset{ %
 language=C++,                        % choose the language of the code
 basicstyle=\footnotesize\ttfamily,
 keywordstyle=\bfseries,
 numbers=left,                      % where to put the line-numbers
 numberstyle=\scriptsize\texttt,    % the size of the fonts that are used for the line-numbers
 stepnumber=1,                      % the step between two line-numbers. If it's 1 each line will be numbered
 numbersep=8pt,                     % how far the line-numbers are from the code
 frame=tb,
 float=htb,
 aboveskip=8mm,
 belowskip=4mm,
 backgroundcolor=\color{cloudwhite},
 showspaces=false,                  % show spaces adding particular underscores
 showstringspaces=false,            % underline spaces within strings
 showtabs=false,                    % show tabs within strings adding particular underscores
 tabsize=2,	                    % sets default tabsize to 2 spaces
 captionpos=b,                      % sets the caption-position to bottom
 breaklines=true,                   % sets automatic line breaking
 breakatwhitespace=false,           % sets if automatic breaks should only happen at whitespace
 escapeinside={\%*}{*)},            % if you want to add a comment within your code
 morekeywords={*,var,template,new}  % if you want to add more keywords to the set
}

\usepackage{amsmath}
\usepackage[algochapter,linesnumberedhidden]{algorithm2e}
\usepackage{booktabs}
\usepackage{tabularx}
\usepackage{array}
\usepackage{multirow}
\usepackage{easylist}
\usepackage{siunitx}
\usepackage{url}
\usepackage{caption}
\usepackage{footnote}
\usepackage{hyperref}
\usepackage[noabbrev,nameinlink]{cleveref}
\usepackage{etoolbox}
\usepackage[nonumberlist,acronym,xindy]{glossaries}

\graphicspath{{figures/}}
\setcounter{tocdepth}{3}

\makeglossaries
%\usepackage[xindy]{imakeidx}
%\makeindex


\newtoggle{showcommittee}
\toggletrue{showcommittee}
