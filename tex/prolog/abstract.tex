\chapter*{Abstract}

Mobile robot platforms capable of operating safely and accurately in dynamic environments can have a multitude of applications, ranging from simple delivery tasks to advanced assembly operations. Besides being very productive when performing repetitive jobs with precision and speed, robots can also act as coworkers, helping humans perform their work more efficiently and thus reducing the overall production costs. These abilities rely heavily on a robust navigation stack, which requires a stable and accurate localization system.

This dissertation describes an efficient, modular, extensible and easy to configure localization system, capable of operating on a wide range of mobile robot platforms and environments. It is able to reliably estimate the global position using feature matching and is capable of achieving high accuracy pose tracking using point cloud registration algorithms. It can use several point cloud sensing devices (such as LIDARs or RGB-D cameras) and requires no artificial landmarks. Moreover, it can update the localization map at runtime and dynamically adjust its operation rate based on the predicted robot velocity in order to use the minimum amount of hardware resources. It also offers a detailed analysis of each pose estimation, providing information about the percentage of registered inliers, the root mean square error of the inliers, the angular distribution of the inliers and outliers, the pose corrections that were performed in relation to the expected position and in case of initial pose estimation it also gives the distribution of the acceptable initial poses, which can be very valuable information for a supervisor when the robot is in ambiguous areas that are very similar in different parts of the known environment.

The ROS implementation was tested in several dynamic indoor environments using two mobile robot platforms equipped with LIDARs and RGB-D cameras. Overall tests in a high end laptop with an Intel Core i7 3630QM processor, 16GB DDR3 of memory and NVIDIA GTX680M graphics card demonstrated high accuracy in complex dynamic environments, with less than 1 cm in translation error and less than 1 degree in rotation error. Execution times ranged from 5 to 30 milliseconds in a 3 DoF setup and from 50 to 150 milliseconds in a full dynamic 6 DoF configuration.



\chapter*{Resumo}

Plataformas móveis robóticas capazes de operar com precisão e de forma segura em ambientes dinâmicos têm um alargado espetro de aplicações, desde simples entregas de objetos até operações complexas de montagem. Para além de serem bastante produtivas e precisas a realizar trabalhos repetitivos, podem também colaborar com humanos para melhorar a produtividade global de uma dada tarefa e assim reduzir os custos de produção. Para atingir estes requisitos de operação é necessário um sistema de navegação robusto, que por sua vez requer um módulo de localização preciso e estável.

Esta dissertação descreve um sistema de localização eficiente, modular, extensível e fácil de configurar, capaz de operar num alargado conjunto de plataformas móveis e ambientes. É capaz de estimar a posição inicial de um robot usando métodos de associação de características geométricas e consegue seguir a sua pose com alta precisão através de algoritmos de registo de nuvens de pontos. A sua implementação consegue tirar partido de vários sensores laser e câmaras RGB-D e não necessita de marcadores artificiais ou modificação do ambiente. Possui ainda a capacidade de atualizar o mapa incrementalmente e ajustar a sua frequência de funcionamento de acordo com a velocidade do robot de forma a usar o mínimo de recursos computacionais. Para facilitar a avaliação da qualidade da localização para operações críticas, cada estimativa da pose do robot é acompanhada com a análise do registo da nuvem de pontos, contendo informação acerca da percentagem de pontos corretamente registados, a raiz quadrada do erro quadrático médio, a distribuição angular dos pontos classificados como pertencentes e não pertencentes ao mapa de referência, as correções aplicadas à estimativa da pose e no caso de ser efetuada localização global também é disponibilizada a distribuição das poses iniciais aceitáveis, o que pode ser informação bastante útil para um supervisor quando o robot está em posições ambíguas do ambiente nas quais existe geometria semelhante em sítios diferentes do mapa.

A implementação em ROS foi testada em vários ambientes dinâmicos recorrendo a duas plataformas móveis equipadas com LIDARs e câmaras RGB-D. Os resultados obtidos num portátil com CPU Intel Core i7 3630QM, 16GB DDR3 de memória e placa gráfica NVIDIA GTX680M demonstraram que o sistema consegue fazer a estimativa da posição do robot com um erro de translação inferior a 1 centímetro e um erro de rotação abaixo de 1 grau. Os tempos de execução oscilaram entre 5 e 30 milissegundos para uma configuração 3 DoF e entre 50 e 150 milissegundos para 6 DoF.
